% \chapter{
% 	۱)بررسی انواع مدل‌های ارائه شده برای دیتابیس‌ها و مزایا و معایب آن‌ها نسبت به مدل رابطه‌ای
% }



\newenvironment{eng}{\begin{LTRbibitems}\resetlatinfont{#1}\end{LTRbibitems}}

\section*{\centering سوال اول
}



می‌دانیم که عملگر
$select$
یک عملگر
$basic$
است.


عملگر
$intersection$
دو عملگر
$select$
را با یکدیگر ترکیب می‌کند، امام تنها سطرهایی را بازمی‌گرداند که در 
$select$
 اول موجود هستند و سطرهایی دقیقا مشابهِ آن‌ها، در
$select$
اول نیز موجود است.


\section*{\centering سوال دوم
}

کوئری اول را بررسی می‌کنیم.
این کوئری، حاصلِ ضربِ کارتزین نتیجه‌ی دو کوئریِ زیر است:




    
% \eng{SELECT A FROM R WHERE B = 7}

\begin{LTRbibitems}\resetlatinfont{SELECT A FROM R WHERE B = 1}\end{LTRbibitems}

\begin{LTRbibitems}\resetlatinfont{SELECT A FROM S WHERE B = 1}\end{LTRbibitems}

% $ SELECT A FROM R WHERE B = 1 $

% $ SELECT C FROM S WHERE B = 1 $

یعنی ابتدا از هر دو جدول، سطرهایی را که مقدار ستون
$B$
مساوی ۱ است، انتخاب می‌کنیم و سپس آن‌ها را با هم ضرب می‌کنیم.
اگر در جدول
$R$
برای مثال، ۲۰ سطر وجود داشته باشد، که از میان این ۲۰ سطر، ۴ سطر باشد که مقدار ستون
$B$
مساوی ۱ است، و در جدول
$S$
۳ سطر باشد که مقدار ستون
$B$
مساوی ۱ است، این کوئری، ۱۲ سطر را باز می‌گرداند.

حال کوئری دوم را بررسی می‌کنیم.
در این کوئری، ابتدا ضرب کارتزین بین نتیجه‌ی این دو کوئری انجام می‌شود:


\begin{LTRbibitems}\resetlatinfont{SELECT A FROM R}\end{LTRbibitems}

\begin{LTRbibitems}\resetlatinfont{SELECT * FROM S WHERE B = 1}\end{LTRbibitems}

یعنی ابتدا از جدول
$R$
همه‌ی سطرها را انتخاب می‌کنیم و سپس از جدول
$S$
سطرهایی را که مقدار ستون
$B$
مساوی ۱ است، انتخاب می‌کنیم.
اگر در جدول
$R$
۴ سطر باشد و در جدول
$S$
۳ سطر باشد، این کوئری، ۱۲ سطر را باز می‌گرداند.

در نهایت، یک  عملگر
$Project$
هم روی نتیجه اعمال شده که ستون‌های
$A$
و
$C$
را بازمی‌گرداند.

بر اساس فرضی که در بخش قبلی مطرح کردیم،‌ در جدول
$R$
۲۰ سطر وجود داشت.
همچنین در جدول
$S$
۳ سطر وجود داشت که مقدار ستون
$B$
مساوی ۱ باشد.
در نتیجه، از ضرب کارتزین این دو، ۶۰ سطر به دست می‌آید.
در نهایت، ستون‌های   
$A$
و
$C$
از این ۶۰ سطر را داریم.
در حالی که در کوئری قبلی، تنها ستون 
$A$
از ۱۲ سطر را داشتیم.
واضح است که این دو کوئری نتایج متفاوتی را بازمی‌گردانند.


حال سومین کوئری را بررسی می‌کنیم.


در این کوئری ابتدا سطرهایی از جدول
$S$
انتخاب شده‌اند که در آن‌ها مقدار ستون
$B$
مساوی ۱ است.


سپس بین حاصل و کل جدول
$R$
عملگر
$Natural Join$
اعمال می‌شود.
این عملگر، ابتدا ستون‌های مشترک را پیدا می‌کند و سپس سطرهایی را از جدول
$R$
انتخاب می‌کند که مقدار ستون‌های مشترک با سطرهای انتخاب شده از جدول
$S$
مساوی باشد.
در این کوئری، ستون مشترک
$B$
است، پس مقادیری بازگردانده می‌شوند که در آن‌ها مقدار
$B$
در هر دو جدول مساوی ۱ باشد.
در نتهایت هم یک 
$Project$
روی نتیجه اعمال می‌شود که ستون‌های
$A$
و
$C$
را باز می‌گرداند.

با فرض‌هایی که در بخش اول در نظر گرفتیم، این کوئری را نیز بررسی می‌کنیم.
از اولین کوئری، تعداد ۳ سطر بازمی‌گردد.

هنگامی که روی این ۳ سطر و کل جدول
$R$
عملگر 
$Natural Join$
اعمال می‌شود، از آن‌جایی که در سه سطر انتخاب شده، مقدار ستون
$B$
مساوی ۱ است، از آن‌جایی که در کل جدول
$R$
هم  در ۴ ستون این شرط برقرار است، نهایتا ۱۲ سطر بازمی‌گردد و سپس ستون‌های
$A$
و
$C$
از آن‌ها باز می‌گردد.
ظاهرا این کوئری، مشابه کوئری اول عمل می‌کند.

حال برای اطمینان، عملگرهای آنان را بررسی می‌کنیم.

در هر دو کوئری، ابتدا سطرهایی که در آن‌ها مقدار
$B$
مساوی ۱ است، از جدول
$S$
انتخاب شده است.
در کوئری اول سپس یک ضرب کارتزین میان این سطرها و سطرهایی از جدول
$R$
انجام می‌شود که مقدار ستون
$B$
در آن‌ها هم مساوی ۱ باشد.

این دقیقا همان اتفاقی است که در کوئری سوم رخ می‌دهد هنگامی که عملگر
$Natural Join$
اعمال می‌شود.



\section*{\centering سوال سوم}

الف)

در کوئری اول، ابتدا تمام سطرهایی که در آن‌ها مقدار 
$salary$
بزرگترمساوی ۹۰۰۰۰ است انتخاب می‌شوند. سپس مقادیر ستون‌های
$name$
و
$dept_name$
از آن‌ها باز می‌گردد.
پس نتیجه‌ی نهایی، جدولی به این شکل خواهد بود:




\begin{LTRbibitems}
    \resetlatinfont{
        
        \begin{tabular}{|c|c|c|}
            \hline
            name & deptname \\ \hline
            Einstein & Physics \\ \hline
            Wu & Finance \\ \hline
            Brandt & Comp. Sci. \\ \hline
            % Ali & CS \\ \hline
            % Reza & CS \\ \hline
            \end{tabular}
    }


\end{LTRbibitems}


اما در کوئری دوم، ابتدا ستون‌های
$deptname$
و
$salary$
با عملگر
$Project$
انتخاب می‌شوند.

سپس کوئری روی آن‌ها زده می‌شود تا سطرهایی که در آن‌ها مقدار
$salary$
بزرگترمساوی ۹۰۰۰۰ است انتخاب شوند.
با توجه به این که در دیتایی که به این کوئری داده می‌شود، دیگر ستون
$salary$
وجود ندارد، نتیجه‌ی نهایی، جدولی بدونِ سطر خواهد بود.



ب)

دیدیم که در کوئری دوم، نتیجه‌ای نداشتیم. پس هنگامی که قرار است عملگر،
$Select$
اعمال شود، باید دقت داشت که از چه فیلدهایی استفاده می‌کند، چون اگر از فیلدی استفاده کنیم که در جدول‌هایی که قبلا انتخاب شده‌اند، وجود نداشته باشد، نتیجه‌ی نهایی، جدولی خالی خواهد بود.
به طور کلی بهتر است ابتدا عملگر
$Select$
اعمال شود و سپس عملگر
$Project$
اعمال شود.



\section*{\centering سوال چهارم}
جدول
$T1$
به این شکل است:


\begin{LTRbibitems}
    \resetlatinfont{
        
        \begin{tabular}{|c|c|}
            \hline
            A & B \\ \hline
            a1 & b1 \\ \hline
            a2 & b2 \\ \hline
            a3 & b3 \\ \hline
        \end{tabular}
}

\end{LTRbibitems}

جدول 
$T2$
به این شکل است:


\begin{LTRbibitems}
    \resetlatinfont{
        
        \begin{tabular}{|c|c|}
            \hline
            B & C \\ \hline
            b1 & c1 \\ \hline
            b1 & c2 \\ \hline
            b2 & c2 \\ \hline
            b2 & c3 \\ \hline
            b2 & c4 \\ \hline
        \end{tabular}
}

\end{LTRbibitems}



الف)

حاصل عملگر
$Natural Join$
بین
$T1$
و
$T2$
از ما خواسته شده است، که به این شکل است:


\begin{LTRbibitems}
    \resetlatinfont{
        
        \begin{tabular}{|c|c|c|}
            \hline
            A & B & C \\ \hline
            a1 & b1 & c1 \\ \hline
            a1 & b1 & c2 \\ \hline
            a2 & b2 & c2 \\ \hline
            a2 & b2 & c3 \\ \hline
            a2 & b2 & c4 \\ \hline
        \end{tabular}

        }
\end{LTRbibitems}


در دومین کوئری
 ابتدا عملگر
$Project$
ستون
$B$
را از جدول‌ها انتخاب می‌کند.
حاصل، این دو ستون خواهند بود:


\begin{LTRbibitems}
    \resetlatinfont{
        
        \begin{tabular}{|c|}
            \hline
            B \\ \hline
            b1 \\ \hline
            b2 \\ \hline
            b3 \\ \hline
        \end{tabular}
    }
\end{LTRbibitems}


\begin{LTRbibitems}
    \resetlatinfont{
        
        \begin{tabular}{|c|}
            \hline
            B \\ \hline
            b1 \\ \hline
            b1 \\ \hline
            b2 \\ \hline
            b2 \\ \hline
            b2 \\ \hline
        \end{tabular}
    }
\end{LTRbibitems}

سپس عملگر
$set intersection$
بین این دو حاصل، اعمال می‌شود و نتیجه‌ی نهایی، به این شکل خواهد بود:


\begin{LTRbibitems}
    \resetlatinfont{
        
        \begin{tabular}{|c|}
            \hline
            B \\ \hline
            b1 \\ \hline
            b2 \\ \hline
        \end{tabular}
    }


\end{LTRbibitems}


ب)

عملگرهای
$Natural Join$
و
$intersection$
کاملا عملگرهای متفاوتی هستند.

با عملگر
$Natural Join$
پس از یافتن سطرهایی که مقدارِ سطرهایشان در ستون‌های مشترک، یکی‌ست، ضرب کارتزین بین این داده‌ها انجام می‌شود.
اما در عملگر
$intersection$
به نوعی اشتراک گرفته می‌شود و سطرهایی که در هر دو جدول دقیقا یکی هستند، بازگردانده می‌شوند.





\section*{\centering سوال پنجم}

جدول
$Student$


\begin{LTRbibitems}
    \resetlatinfont{
        
        \begin{tabular}{|c|c|c|}
            \hline
            ID & Name & Nationality \\ \hline
            1 & Jon Snow & USA \\ \hline
            2 & Jame Bond & UK \\ \hline
            3 & Winston Churchill & USA \\ \hline
            4 & John F. Kennedy & USA \\ \hline
            5 & Jakie Chan & China \\ \hline
            6 & Richard White & USA \\ \hline
            7 & Bruce Lee & USA \\ \hline
            8 & Hugo Lafayette & France \\ \hline
            9 & Ben Kenobi & USA \\ \hline
            10 & Harry Potter & UK \\ \hline
            11 & Son Goku & Japan \\ \hline 
            12 & Wonder Woman & UK \\ \hline
            13 & Sun Tzu & China \\ \hline
            14 & Tony Stark & USA \\ \hline
            15 & Leia Organa & USA \\ \hline
            
        \end{tabular}
    }
\end{LTRbibitems}


جدول
$Enrollment$
به این شکل است:


\begin{LTRbibitems}
    \resetlatinfont{
        
        \begin{tabular}{|c|c|c|c|c|}
            \hline
            StudentID & CourseID & Grade & SectionNum & GroupID \\ \hline
            1  & CS448 & A & 2 & 3 \\ \hline
            4  & CS448 & A & 1 & 2 \\ \hline
            5  & CS448 & B & 1 & 1 \\ \hline
            6  & CS448 & A & 1 & 1 \\ \hline
            9  & CS448 & B & 2 & 3 \\ \hline
            10 & CS448 &  A & 2 & 4 \\ \hline
            11 & CS448 &  C & 2 & 4 \\ \hline
            12 & CS448 &  A & 2 & 3 \\ \hline
            13 & CS448 &  A & 1 & 1 \\ \hline
            2  & CS580 & A & 1 & 1 \\ \hline
            3  & CS580 & A & 1 & 1 \\ \hline
            4  & CS580 & B & 1 & 2 \\ \hline
            6  & CS580 & A & 1 & 2 \\ \hline
            8  & CS580 & A & 1 & 2 \\ \hline
            10 & CS580 &  A & 1 & 1 \\ \hline
            12 & CS580 &  B & 2 & 3 \\ \hline
            15 & CS580 &  A & 2 & 3 \\ \hline

        \end{tabular}
    }
\end{LTRbibitems}




جدول
$Course$
به این شکل است:




\begin{LTRbibitems}
    \resetlatinfont{
        
        \begin{tabular}{|c|c|c|}
            \hline
            CourseID & InstructorID & Name \\ \hline
            CS448 & 7 & Introduction to Relational Database Systems \\ \hline
            CS390 & 2 & Linear Algebra \\ \hline
            CS580 & 14 & Algorithm Design, Analysis, And Implementation \\ \hline

        \end{tabular}
    }
\end{LTRbibitems}






