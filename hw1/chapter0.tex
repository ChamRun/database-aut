% \chapter{مقدمه}
% {مصور سازی}\LTRfootnote{Visualization}
%  داده‌های با ابعاد بالا مسئله‌ی مهمی در بسیاری از دامنه‌ها می‌باشد. این مسئله با بازه‌ی بزرگی از ابعاد (از سی بعد برای داده‌های پزشکی تا ده‌ها هزار بعد برای داده‌های سند‌ها) درگیر است.
% \\
% در دهه‌ی اخیر راه‌حل‌های متنوعی برای این مسئله ارائه شده است
% \cite{lee2007nonlinear}
%  که می‌توان به روش‌های شمایل نگاری، روش‌های براساس
% {پیکسل}\LTRfootnote{Pixel}
% و روش‌هایی که ابعاد را به صورت رئوس گراف نشان می‌دهند اشاره کرد.
% \\
% اکثر این روش‌ها ابزاری ارائه می‌دهند که داده‌ها را بتوان به صورت دو بعدی نمایش داد و تحلیل را به مشاهده انسانی واگذار می‌کنند.
% این موضوع باعث می‌شود که کاربرد این روش‌ها در دنیای واقعی برای داده‌هایی که ده‌ها هزار بعد دارند کم شود.
% \\
% برخلاف روش‌های بالا، روش‌های کاهش بعد سعی می‌کنند داده‌های با ابعاد بالا را به داده‌های دو یا سه بعدی خلاصه کنند.
% \\
% روش‌های خطی و سنتی‌ای مانند
% {تجزیه و تحلیل اجزای اصلی یا پی ‌سی‌ ای}\LTRfootnote{PCA}
% یا 
% {مقایسه چند بعدی کلاسیک یا ام دی اس}\LTRfootnote{MDS}
% روش‌هایی هستند که سعی دارند داده‌های غیرمشابه را پس از نگاشت تا جای ممکن از هم دور نگه دارند. در روش‌های غیرخطی سعی می‌شود که داده‌هایی که در توصیف اصلی خود به همدیگر نزدیک هستند،‌ پس از نگاشت و خلاصه شدن نیز باز هم به همدیگر نزدیک بمانند،‌ که پیاده‌سازی این مهم توسط روش‌های خطی امکان پذیر نمی‌باشد.
% \\
% روش‌های غیرخطی زیادی برای کاهش بعد ارائه شده است که سعی دارند ساختار محلی داده را نگه دارند. از این روش‌ها می‌توان به
% نگاشت سامون،
% {تحلیل و تجزیه اجزای منحنی یا سی سی ای}\LTRfootnote{CCA}
% ،
% {تعبیه همسایه تصادفی یا اس ان ای}\LTRfootnote{SNE}
% \cite{hinton2002stochastic}
% ،
% {حداکثر واریانس آشکار یا ام وی یو}\LTRfootnote{MVU}
% ،
% {تعبیه خطی محلی یا ال ال ای}\LTRfootnote{LLE}
% و
% {نگاشت ویژه لاپلاس}\LTRfootnote{Laplacian Eigenmaps}
% اشاره کرد.
% با اینکه روش‌های بر داده‌های مصنوعی با ابعاد بالا بسیار خوب عمل می‌کنند اما در مصور سازی داده‌های با ابعاد بالا، نمی‌توانند عملکرد مناسبی داشته باشند.
% اکثر این روش‌ها نمی‌توانند ساختار عمومی و محلی داده‌ها را در یک نگاشت به خوبی مصور کنند.
% برای مثال، یک مدل ام وی یو با نظارت متوسط نمی‌تواند رقم‌های با دست نوشته را به خوشه‌های طبیعی هر رقم نگاشت کند و رقم‌ها را از هم جدا کند.
% \\
% در این گزارش ما به بررسی روشی می‌پردازیم که داده‌های با ابعاد بالا را به ماتریس شباهت داده‌ها تبدیل می‌کند و سپس با روشی به نام
% {تی اس ان ای}\LTRfootnote{t-SNE}
% \cite{van2008visualizing}
% \cite{rauber2016visualizing}
% این ماتریس به دست آمده را مصور سازی می‌کند.
% \\
% روش تی اس ان ای توانایی نمایش دادن اکثر ساختار محلی داده‌ها را داشته و در کنار آن ساختار عمومی داده‌ها مانند خوشه‌های موجود در داده‌ها را نیز نمایش می‌دهد.
% \\
% در این گزارش ما عملکر تی اس ان ای را با مقایسه کردن آن با هفت روش بیان شده بر روی پنج مجموعه داده‌ی بدست آمده از دامنه‌های مختلف بررسی خواهیم کرد، نتایج نشان می‌دهد که این روش در اکثر دامنه‌ها نسبت به بقیه روش‌ها دارای برتری می‌باشد.
% \\
% در فصل بعدی اس ان ای را معرفی می‌کنیم که مفاهیم آن پایه تی اس ان ای می‌باشد، سپس در فصل بعدی آن تی اس ان ای را معرفی می‌کنیم که دو تفاوت اساسی با اس ان ای دارد، در فصل بعد آن شرایط آزمایش و نتایج آزمایش‌ها را بیان کرده و در فصل بعد آن نشان می‌دهیم که تی اس ان ای چطور می‌تواند تغییر پیدا کند به شکلی که داده‌هایی که ابعاد‌ آن‌ها بسیاری بیشتر از ده هزار بعد است را نمایش دهد و در فصل بعد آن نتایج آزمایش‌ها با دقت بیشتری بررسی می‌شود، در فصل آخر نیز جمع بندی انجام می‌شود.